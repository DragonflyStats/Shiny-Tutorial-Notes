
\documentclass[a4paper,12pt]{article}
%%%%%%%%%%%%%%%%%%%%%%%%%%%%%%%%%%%%%%%%%%%%%%%%%%%%%%%%%%%%%%%%%%%%%%%%%%%%%%%%%%%%%%%%%%%%%%%%%%%%%%%%%%%%%%%%%%%%%%%%%%%%%%%%%%%%%%%%%%%%%%%%%%%%%%%%%%%%%%%%%%%%%%%%%%%%%%%%%%%%%%%%%%%%%%%%%%%%%%%%%%%%%%%%%%%%%%%%%%%%%%%%%%%%%%%%%%%%%%%%%%%%%%%%%%%%
\usepackage{eurosym}
\usepackage{vmargin}
\usepackage{amsmath}
\usepackage{graphics}
\usepackage{epsfig}
\usepackage{subfigure}
\usepackage{fancyhdr}
%\usepackage{listings}
\usepackage{framed}
\usepackage{graphicx}

\setcounter{MaxMatrixCols}{10}
%TCIDATA{OutputFilter=LATEX.DLL}
%TCIDATA{Version=5.00.0.2570}
%TCIDATA{<META NAME="SaveForMode" CONTENT="1">}
%TCIDATA{LastRevised=Wednesday, February 23, 2011 13:24:34}
%TCIDATA{<META NAME="GraphicsSave" CONTENT="32">}
%TCIDATA{Language=American English}

\pagestyle{fancy}
\setmarginsrb{20mm}{0mm}{20mm}{25mm}{12mm}{11mm}{0mm}{11mm}
\lhead{Shiny} \rhead{Dublin \texttt{R}}
\chead{Reactivity}
%\input{tcilatex}
\begin{document}

\subsection*{Layout}

\begin{itemize}
\item To create a display with a fluid, unbroken layout, Shiny \texttt{ui.R} scripts need the function \texttt{fluidPage}. Shiny knows where to put your app’s elements when it reads them in the \texttt{fluidPage} function.

\item The following \texttt{ui.R} script creates a user-interface that has a title panel, a sidebar panel, and a main panel. Note that these elements are placed within the fluidPage function.

\begin{framed}
\begin{verbatim}
# ui.R

shinyUI(fluidPage(
  titlePanel("title panel"),

  sidebarLayout(
    sidebarPanel( "sidebar panel"),
    mainPanel("main panel")
  )
))
\end{verbatim}
\end{framed}


\item \texttt{titlePanel} and \texttt{sidebarLayout} are the two most popular elements to add to \texttt{fluidPage}. They create a basic Shiny app with a sidebar.

\subitem \texttt{sidebarLayout} always takes two arguments:
\subitem \texttt{sidebarPanel} function output
\subitem \texttt{mainPanel }function output

\item These functions place content in either the sidebar or the main panels.
\item 
By default the sidebar appears on the left side of your app’s display. To move the sidebar to the right, in \texttt{sidebarLayout} set position to “right.”
\end{itemize}
\begin{framed}
\begin{verbatim}
# ui.R

shinyUI(fluidPage(
  titlePanel("title panel"),

  sidebarLayout(position = "right",   #<- HERE
    sidebarPanel( "sidebar panel"),
    mainPanel("main panel")
  )
))
\end{verbatim}
\end{framed}

\end{document}
