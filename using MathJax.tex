The function \texttt{withMathJax()} is a wrapper function to load the MathJax library in a shiny app. 

For static HTML content, we only need to call \texttt{withMathJax()}  once. 

However, for dynamic UI output via \texttt{renderUI()}, we must wrap the content that contains math expressions in withMathJax(), 
because we have to call theMathJax function \texttt{MathJax.Hub.Queue(["Typeset", MathJax.Hub])} to render math manually, 
which is what \texttt{withMathJax()} does.

%==================================================================%
\begin{framed}
\begin{verbatim}
library(shiny)

shinyServer(function(input, output, session) {
  output$ex1 <- renderUI({
    withMathJax(helpText('Dynamic output 1:  $$\\alpha^2$$'))
  })
  output$ex2 <- renderUI({
    withMathJax(
      helpText('and output 2 $$3^2+4^2=5^2$$'),
      helpText('and output 3 $$\\sin^2(\\theta)+\\cos^2(\\theta)=1$$')
    )
  })
  output$ex3 <- renderUI({
    withMathJax(
      helpText('The busy Cauchy distribution
               $$\\frac{1}{\\pi\\gamma\\,\\left[1 +
               \\left(\\frac{x-x_0}{\\gamma}\\right)^2\\right]}\\!$$'))
  })
  output$ex4 <- renderUI({
    invalidateLater(5000, session)
    x <- round(rcauchy(1), 3)
    withMathJax(sprintf("If \\(X\\) is a Cauchy random variable, then
                        $$P(X \\leq %.03f ) = %.03f$$", x, pcauchy(x)))
  })
  output$ex5 <- renderUI({
    if (!input$ex5_visible) return()
    withMathJax(
      helpText('You do not see me initially: $$e^{i \\pi} + 1 = 0$$')
    )
  })
})
\end{verbatim}
\end{framed}
%==================================================================%
\subsection*{IU side}

\begin{framed}
\begin{verbatim}
library(shiny)

shinyUI(fluidPage(
  title = 'MathJax Examples',
  withMathJax(),
  helpText('An irrational number \\(\\sqrt{2}\\)
           and a fraction $$1-\\frac{1}{2}$$'),
  helpText('and a fact about \\(\\pi\\):
           $$\\frac2\\pi = \\frac{\\sqrt2}2 \\cdot
           \\frac{\\sqrt{2+\\sqrt2}}2 \\cdot
           \\frac{\\sqrt{2+\\sqrt{2+\\sqrt2}}}2 \\cdots$$'),
  uiOutput('ex1'),
  uiOutput('ex2'),
  uiOutput('ex3'),
  uiOutput('ex4'),
  checkboxInput('ex5_visible', 'Show Example 5', FALSE),
  uiOutput('ex5\end{verbatim}
\end{framed}
%==================================================================%
\end{document}
