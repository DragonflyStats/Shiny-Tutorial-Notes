
\documentclass[a4paper,12pt]{article}
%%%%%%%%%%%%%%%%%%%%%%%%%%%%%%%%%%%%%%%%%%%%%%%%%%%%%%%%%%%%%%%%%%%%%%%%%%%%%%%%%%%%%%%%%%%%%%%%%%%%%%%%%%%%%%%%%%%%%%%%%%%%%%%%%%%%%%%%%%%%%%%%%%%%%%%%%%%%%%%%%%%%%%%%%%%%%%%%%%%%%%%%%%%%%%%%%%%%%%%%%%%%%%%%%%%%%%%%%%%%%%%%%%%%%%%%%%%%%%%%%%%%%%%%%%%%
\usepackage{eurosym}
\usepackage{vmargin}
\usepackage{amsmath}
\usepackage{graphics}
\usepackage{epsfig}
\usepackage{subfigure}
\usepackage{fancyhdr}
%\usepackage{listings}
\usepackage{framed}
\usepackage{graphicx}

\setcounter{MaxMatrixCols}{10}
%TCIDATA{OutputFilter=LATEX.DLL}
%TCIDATA{Version=5.00.0.2570}
%TCIDATA{<META NAME="SaveForMode" CONTENT="1">}
%TCIDATA{LastRevised=Wednesday, February 23, 2011 13:24:34}
%TCIDATA{<META NAME="GraphicsSave" CONTENT="32">}
%TCIDATA{Language=American English}

\pagestyle{fancy}
\setmarginsrb{20mm}{0mm}{20mm}{25mm}{12mm}{11mm}{0mm}{11mm}
\lhead{Shiny} \rhead{Dublin \texttt{R}}
\chead{Reactive and Rendering Functions}
%\input{tcilatex}
\begin{document}

\subsection*{Basics of Shiny}

\begin{itemize}

\item To create an shiny app, create a subfolder with the name of the shiy app in your working directory (\texttt{Appname}).

\item There are two components (i.e. script files) of a shiny app.

\item These are called (and must be called) \texttt{ui.R} and
 \texttt{server.R}. 
Also they must be also in the same subfolder.

\item Input elements are defined in \texttt{ui.R}  and processed by \texttt{server.R}, which sends to \texttt{ui.R}.

\item When the Shiny Package is ready, you simply type \texttt{runApp("Appname")}.

% Make sure the working directory is the same folder.
\end{itemize}

%-----------------------------------------------------------------------%
\subsection*{ui.R for a minimal example}
\begin{framed}
\begin{verbatim}

library(shiny)

#OLD LAYOUT - REPLACING WITH "fluidpage"

shinyUI(pageWithSidebar(

.......


)
\end{verbatim}
\end{framed}
%-----------------------------%
\subsection*{Components of this layout}
\begin{itemize}
\item \texttt{sidebarPanel()}
\item \texttt{headerPanel()}
\item \texttt{mainPanel()}
\end{itemize}

%-----------------------------------------------------------------------%

\begin{framed}
\begin{verbatim}

library(shiny)

shinyUI(pageWithSidebar(

headerPanel(........),
sidebarPanel(........),
mainPanel(.........)

)
\end{verbatim}
\end{framed}

%-----------------------------------------------------------------------%
\subsection*{Tabbed Panels}
\begin{framed}
\begin{verbatim}

.........
mainPanel(
  tabsetPanel(
     tabPanel(..........),
     tabPanel(..........),
     tabPanel(..........)
  )
)
........
\end{verbatim}
\end{framed}

\newpage
%-----------------------------%
\subsection*{server.R for a minimal example}

%\begin{itemize}
%\item \texttt{shinyServer(....\{...\})}}
% defines the shiny server
%\end{itemize}

\begin{framed}
\begin{verbatim}

library(shiny)

shinyServer(function(input,output){

.......


})
\end{verbatim}
\end{framed}
\end{document}
