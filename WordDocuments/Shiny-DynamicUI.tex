\documentclass{beamer}

\usepackage{amsmath}
\usepackage{framed}
\usepackage{amssymb}

\begin{document}
%-----------------------------------------------------------------------------------------------%
\begin{frame}
\frametitle{Dynamic UI}
\Large
\begin{itemize}
\item Shiny apps are often more than just a fixed set of controls that affect a fixed set of outputs. \item Inputs may need to be shown or hidden depending on the state of another input, or input controls may need to be created on-the-fly in response to user input.
\item Shiny currently has three different approaches you can use to make your interfaces more dynamic. From easiest to most difficult, they are:
\end{itemize}
\end{frame}
%-----------------------------------------------------------------------------------------------%

%-----------------------------------------------------------------------------------------------%
\begin{frame}
\frametitle{Dynamic UI}
\Large
\begin{enumerate}
\item The \texttt{conditionalPanel} function, which is used in \texttt{ui.R} and wraps a set of UI elements that need to be dynamically shown/hidden
\item The \texttt{renderUI} function, which is used in \texttt{server.R} in conjunction with the \texttt{htmlOutput} function in \texttt{ui.R}, lets you generate calls to UI functions and make the results appear in a predetermined place in the UI.

\item Use \textit{\textbf{JavaScript}} to modify the webpage directly.
\end{enumerate}
%Let’s take a closer look at each approach.
\end{frame}

%-----------------------------------------------------------------------------------------------%
\begin{frame}[fragile]
\frametitle{Dynamic UI}
\Large
\textbf{1. \texttt{conditionalPanel}}\\
\begin{itemize}
\item \texttt{conditionalPanel} creates a panel that shows and hides its contents depending on the value of a JavaScript expression. \item Even if you don’t know any JavaScript, simple comparison or equality operations are extremely easy to do, as they look a lot like \texttt{R} (and many other programming languages).
\end{itemize}
\end{frame}

\begin{frame}
\frametitle{Dynamic UI}
\Large
\vspace{-1cm}
\textbf{1. \texttt{conditionalPanel}}\\
\begin{itemize}
\item Here’s an example for adding an optional smoother to a \textit{\textbf{ggplot}}, and choosing its smoothing method (next slide)
\end{itemize}
\end{frame}

\begin{frame}[fragile]
\frametitle{Dynamic UI}
(Example: \textbf{ex11})
\begin{framed}
\begin{verbatim}
# Partial example
checkboxInput("smooth", "Smooth"),
conditionalPanel(
  condition = "input.smooth == true",
  selectInput("smoothMethod", "Method",
      list("lm", "glm", "gam", "loess", "rlm"))
)
\end{verbatim}
\end{framed}
\end{frame}

%-----------------------------------------------------------------------------------------------%
\begin{frame}
\frametitle{Dynamic UI}
\Large
\textbf{1. \texttt{conditionalPanel}}\\
\begin{itemize}
\item In this example, the select control for \texttt{smoothMethod} will appear only when the smooth checkbox is checked. 
\item Its condition is "\texttt{input.smooth == true}", which is a JavaScript expression that will be evaluated whenever any inputs/outputs change.
\end{itemize}
\end{frame}
%-----------------------------------------------------------------------------------------------%
\begin{frame}
\frametitle{Dynamic UI}
\Large
\textbf{1. \texttt{conditionalPanel}}\\
\begin{itemize}
\item The condition can also use output values; they work in the same way (\texttt{output.foo} gives you the value of the output foo). \item If you have a situation where you wish you could use an \texttt{R} expression as your condition argument, you can create a reactive expression in \texttt{server.R} and assign it to a new output, then refer to that output in your condition expression. 
\end{itemize}

%For example:
\end{frame}

%-----------------------------------------------------------------------------------------------%
\begin{frame}[fragile]
\textbf{ui.R} \\
(N.B \texttt{output.nrows})

\begin{framed}
\begin{verbatim}
# Partial example
selectInput("dataset", "Dataset", 
   c("diamonds", "rock", "pressure", "cars")),
   
conditionalPanel(
  condition = "output.nrows",  #<-----THIS!!!
  checkboxInput("headonly", 
  "Only use first 1000 rows"))
\end{verbatim} 
\end{framed}
\end{frame}
\begin{frame}[fragile] 
\textbf{server.R}
\begin{framed}
\begin{verbatim}
# Partial example
datasetInput <- reactive({
   switch(input$dataset,
          "rock" = rock,
          "pressure" = pressure,
          "cars" = cars)
})


output$nrows <- reactive({
  nrow(datasetInput())        #.....COMES FROM HERE
})
\end{verbatim}
\end{framed}
\end{frame}
%-----------------------------------------------------------------------------------------------%
\begin{frame}
\frametitle{Dynamic UI}
\Large
\textbf{1. \texttt{conditionalPanel}}
\vspace{1cm}
\begin{itemize}
\item\textbf{However......} this technique requires server-side calculation (which could take a long time, depending on what other reactive expressions are executing) 
\item RStudio recommend that you avoid using output in your conditions unless absolutely necessary.
\end{itemize}
\end{frame}
%-----------------------------------------------------------------------------------------------%
\begin{frame}
\frametitle{Dynamic UI}
\Large
\textbf{2. \texttt{renderUI}}

\begin{itemize}
\item \textbf{Note:} This feature should be considered experimental. Let us know whether you find it useful.

\item Sometimes it’s just not enough to show and hide a fixed set of controls. Imagine prompting the user for a latitude/longitude, then allowing the user to select from a checklist of cities within a certain radius. In this case, you can use the renderUI expression to dynamically create controls based on the user’s input.
\end{itemize}
\end{frame}
%-----------------------------------------------------------------------------------------------%
\begin{frame}[fragile]
\textbf{ui.R- Partial example}
\begin{framed}
\begin{verbatim}
# Partial example
numericInput("lat", "Latitude"),
numericInput("long", "Longitude"),
uiOutput("cityControls")
\end{verbatim}  
\end{framed}
\textbf{server.R - Partial example}
\begin{framed}
\begin{verbatim}
output$cityControls <- renderUI({
  cities <- getNearestCities(input$lat, input$long)
  checkboxGroupInput("cities", "Choose Cities", 
  cities)
})
\end{verbatim}
\end{framed}
\end{frame}

%-----------------------------------------------------------------------------------------------%
\begin{frame}
\frametitle{Dynamic UI}
\Large
\textbf{2. \texttt{renderUI}}

\begin{itemize}
\item \texttt{renderUI} works just like \texttt{renderPlot}, \texttt{renderText}, and the other output rendering functions you’ve seen before, but it expects the expression it wraps to return an HTML tag (or a list of HTML tags, using tagList). These tags can include inputs and outputs.

\item In \texttt{ui.R}, use a \texttt{uiOutput} to tell Shiny where these controls should be rendered.
\end{itemize}
\end{frame}
%-----------------------------------------------------------------------------------------------%
\begin{frame}

\frametitle{Dynamic UI}
\Large
\textbf{3. \texttt{JavaScript}}\\

\textit{Comments on Shiny's Tutorial Page}
\begin{itemize}
\item \textbf{Note:} This feature should be considered experimental. Let us know whether you find it useful.

\item JavaScript/jQuery can be used to modify the page directly. General instructions for doing so are outside the scope of the Shiny tutorial, except to mention an important additional requirement. 
\end{itemize}
\end{frame}
%-----------------------------------------------------------------------------------------------%
\begin{frame}
Each time you add new inputs/outputs to the DOM, or remove existing inputs/outputs from the DOM, you need to tell Shiny. Our current recommendation is:
\begin{itemize}
\item Before making changes to the DOM that may include adding or removing Shiny inputs or outputs, call \texttt{Shiny.unbindAll()}.
\item After such changes, call \texttt{Shiny.bindAll()}.
\end{itemize}If you are adding or removing many inputs/outputs at once, it’s fine to call Shiny.unbindAll() once at the beginning and \texttt{Shiny.bindAll()} at the end – it’s not necessary to put these calls around each individual addition or removal of inputs/outputs.
\end{frame}
%-----------------------------------------------------------------------------------------------%

\end{document}
