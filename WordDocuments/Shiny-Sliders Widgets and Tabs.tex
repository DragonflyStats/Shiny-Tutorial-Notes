\documentclass{beamer}

\usepackage{amsmath}
\usepackage{framed}
\usepackage{amssymb}

\begin{document}
%------------------------------------- %
\begin{frame}
\frametitle{Sliders Widgets and Tabs}
Customizing Sliders
Shiny slider controls are extremely capable and customizable. Features supported include:

The ability to input both single values and ranges
Custom formats for value display (e.g for currency)
The ability to animate the slider across a range of values
Slider controls are created by calling the sliderInput function. The ui.R file demonstrates using sliders with a variety of options:
\end{frame}
%------------------------------------- %
%------------------------------------- %
\begin{frame}
\frametitle{Sliders}
Server Script
The server side of the Slider application is very straightforward: it creates a data frame containing all of the input values and then renders it as an HTML table:
\end{frame}
%------------------------------------- %
%------------------------------------- %
\begin{frame}[fragile]
\frametitle{Widgets}
ex10
\begin{verbatim}
 sidebarPanel(
    selectInput("dataset", "Choose a dataset:", 
                choices = c("rock", "pressure", "cars")),

    numericInput("obs", "Number of observations to view:", 10),

    helpText("Note: while the data view will show only the specified",
             "number of observations, the summary will still be based",
             "on the full dataset."),

    submitButton("Update View")
  )
\end{verbatim}
\end{frame}
%------------------------------------- %
%------------------------------------- %
\begin{frame}
\frametitle{Widgets}

\end{frame}
%------------------------------------- %
%------------------------------------- %
\begin{frame}
\frametitle{Tab Panels}

Tabsets are created by calling the tabsetPanel function with a list of tabs created by the tabPanel function. Each tab panel is provided a list of output elements which are rendered vertically within the tab.

In this example we updated our Hello Shiny application to add a summary and table view of the data, each rendered on their own tab. Here is the revised source code for the user-interface:
\end{frame}
%------------------------------------- %
%------------------------------------- %
\begin{frame}[fragile]
\frametitle{Widgets}
ex11
\begin{framed}
\begin{verbatim}
mainPanel(
    tabsetPanel(
      tabPanel("Plot", plotOutput("plot")), 
      tabPanel("Summary", verbatimTextOutput("summary")), 
      tabPanel("Table", tableOutput("table"))
    )
  )
\end{verbatim}
\end{framed}
\end{frame}
%------------------------------------- %
%------------------------------------- %
\begin{frame}
\frametitle{Tabs}

\end{frame}
%------------------------------------- %
%------------------------------------- %
\begin{frame}
\frametitle{Tabs}

\end{frame}
%------------------------------------- %
\end{document}