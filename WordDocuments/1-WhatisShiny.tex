Shiny is a tool which is provided by the makers of RStudio to enable the creation of dynamic user interfaces to the 
R programming language.  It includes a rich collection of widgets which the app creator can choose from to allow people
with no coding experience to easily use an R program.  If you don’t want to use the standard widgets 
Shiny also allows you to embed HTML or Javascript in the app, creating a fully customised UI for your R code.
 
 \newpage
There are two main parts to a Shiny app, the UI and the server.  
The server is the portion of the code which deals with the back-end of the application and is where the reactivity of 
the application can be controlled.  In the UI section the interface is defined for the user by combining widgets 
to allow the user to control the server.  These two sections communicate through input and output variables; the user 
controls the input variables through control widgets, and the server creates the output variables which are rendered in the UI.

%-------------------------------%
 
Once your app is complete you can share it using the source files or upload it to GitHub.  
Alternatively you can host the application as a webpage using ShinyApps.io or on your own server using Shiny Server, 
which means people can use your app without having to touch R.  Shiny is available at shiny.rstudio.com.
